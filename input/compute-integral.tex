\documentclass{article}
\usepackage{graphicx} % Required for inserting images
\usepackage{amsmath}
\usepackage{amssymb}
\title{ComputeIntegral}
\author{haiml76 }
\date{December 2025}

\begin{document}

\maketitle
In all the following, we denote by $v=v_p$ the $p$-adic Haar measure.

\section{Computation of the integral}
We want to integrate the function $|\det{M}|_p^s=|\lambda_1^4\lambda_2^6\lambda_3^6\lambda_4^4|_p^{s}$ over the monoid $G^+(\mathbb{Q}_p)$.
We denote by $v_1=v_p(\lambda_1),v_2=v_p(\lambda_2),v_3=v_p(\lambda_3),v_4=v_p(\lambda_4)$ the $p$-adic valuations of the elements on the main diagonal of $M_{11}$ the top-left block of the product matrix $M$.

We denote by $t=p^{-s}$, where $s$ is the complex parameter of our zeta-function.

Thus, $|\lambda_1\lambda_2\lambda_3\lambda_4|_p^s=|\lambda_1|_p^s|\lambda_2|_p^s|\lambda_3|_p^s|\lambda_4|_p^s=(p^{-v_1})^s(p^{-v_2})^s(p^{-v_3})^s(p^{-v_4})^s=(p^{-s})^{v_1}(p^{-s})^{v_2}(p^{-s})^{v_3}(p^{-s})^{v_4}=t^{v_1}t^{v_2}t^{v_3}t^{v_4}$.

Now we need to define our domain of integration, and calculate its Haar measure. Since we are computing a $p$-adic integral, we multiply the function to integrate by the domain measure, the result being our desired integral. 
\subsection{Computation of constraints on $c_1,c_2,c_3,c_4$}
We compute the constraints on $c_1,c_2,c_3,c_4$ by considering the following product matrix elements:
\begin{flalign*}
m_{1,10}=c_1\lambda_2\lambda_3\lambda_1\lambda_1\in\mathbb{Z}_p\iff{v(c_1\lambda_2\lambda_3\lambda_1\lambda_1)=v(c_1)+v(\lambda_1)+v(\lambda_2)+v(\lambda_3)+v(\lambda_4)\geq{0}}.&&\\
m_{2,10}=c_2\lambda_2\lambda_3\lambda_1\lambda_1\in\mathbb{Z}_p\iff{v(c_2\lambda_2\lambda_3\lambda_1\lambda_1)=v(c_2)+v(\lambda_1)+v(\lambda_2)+v(\lambda_3)+v(\lambda_4)\geq{0}}.&&\\
m_{3,10}=c_3\lambda_2\lambda_3\lambda_1\lambda_1\in\mathbb{Z}_p\iff{v(c_3\lambda_2\lambda_3\lambda_1\lambda_1)=v(c_3)+v(\lambda_1)+v(\lambda_2)+v(\lambda_3)+v(\lambda_4)\geq{0}}.&&\\
m_{4,10}=c_4\lambda_2\lambda_3\lambda_1\lambda_1\in\mathbb{Z}_p\iff{v(c_4\lambda_2\lambda_3\lambda_1\lambda_1)=v(c_4)+v(\lambda_1)+v(\lambda_2)+v(\lambda_3)+v(\lambda_4)\geq{0}}.&&
\end{flalign*}
Hence we obtain the following constraints:
\[
v(c_1),v(c_2),v(c_3),v(c_4)\geq-(v_1+v_2+v_3+v_4)\Rightarrow\mu(c_1),\mu(c_2),\mu(c_3),\mu(c_4)=p^{v_1+v_2+v_3+v_4}.\]
Hence our integral is
\[
\int_{\underline\lambda,\underline{a},\underline{b}}t^{4v_1+6v_2+6v_3+4v_4}p^{4v_1+4v_2+4v_3+4v_4}d\mu(\underline\lambda)d\mu(\underline{a})d\mu(\underline{b}).
\]
\subsection{Computation of constraints on the $a,b$ elements}
We consider the following product matrix elements:
\begin{flalign*}
m_{18}=b_{11}\lambda_1\lambda_2\lambda_3\in\mathbb{Z}_p\Rightarrow{v(b_{11})\geq-(v_1+v_2+v_3)}.&&\\
m_{19}=b_{12}\lambda_2\lambda_3\lambda_4\in\mathbb{Z}_p\Rightarrow{v(b_{12})\geq-(v_2+v_3+v_4)}.&&\\
m_{29}=b_{22}\lambda_2\lambda_3\lambda_4\in\mathbb{Z}_p\Rightarrow{v(b_{22})\geq-(v_2+v_3+v_4)}.&&\\
m_{38}=b_{31}\lambda_1\lambda_2\lambda_3\in\mathbb{Z}_p\Rightarrow{v(b_{31})\geq-(v_2+v_3+v_4)}.&&\\
m_{48}=b_{41}\lambda_1\lambda_2\lambda_3\in\mathbb{Z}_p\Rightarrow{v(b_{41})\geq-(v_2+v_3+v_4)}.&&\\
\end{flalign*}
But $b_{11}$ appears also in the following expression:

$m_{49}=(a_{11}a_{22}-b_{11})\lambda_2\lambda_3\lambda_4\in\mathbb{Z}_{p}.$

Hence we have:
\begin{flalign*}
b_{11}\in{p^{-(v_1+v_2+v_3)}}.\\
a_{11}a_{22}-b_{11}\in{p^{-(v_2+v_3+v_4)}}\Rightarrow{b_{11}\in{a_{11}a_{22}+p^{-(v_2+v_3+v_4)}}}.
\end{flalign*}
Which means that $b_{11}\in{p^{-(v_1+v_2+v_3)}\mathbb{Z}_p}\cap\left(a_{11}a_{22}+p^{-(v_2+v_3+v_4)}\mathbb{Z}_p\right)$.

We compute $\mu\left(p^{-(v_1+v_2+v_3)}\mathbb{Z}_p\cap\left(a_{11}a_{22}+p^{-(v_2+v_3+v_4)}\mathbb{Z}_p\right)\right)$, 

the Haar measure of the intersection.

Denote:

$a=a_{11}a_{22}$.

$\alpha=v_1+v_2+v_3$.

$\beta=v_2+v_3+v_4$.

We have the following cases:
\begin{enumerate}
    \item $\alpha\geq\beta$ and $\alpha\geq-v_p(a)$.

Let $x\in{\left(a+p^{-\beta}\mathbb{Z}_p\right)\cap{p^{-\alpha}\mathbb{Z}_p}}$ be some element in the intersection, but if $x\in{a+p^{-\beta}\mathbb{Z}_p}$ then $x=a+b$, where $b\in{p^{-\beta}\mathbb{Z}_p}\iff{v_p(b)\geq{-\beta}}$, but then $v_p(x)\geq\min\{v_p(a),v_p(b)\}\geq\min\{v_p(a),-\beta\}$.

But $\alpha\geq-v_p(a)\iff{v_p(a)\geq-\alpha}\Rightarrow\min\{v_p(a),-\beta\}\geq\min\{-\alpha,-\beta\}$,

and $\alpha\geq\beta\iff-\beta\geq-\alpha\Rightarrow{v_p(x)\geq\min\{-\alpha,-\beta\}=-\alpha}\Rightarrow{x\in{p^{-\alpha}\mathbb{Z}_p}}$.

But all this shows that $\left(a+p^{-\beta}\mathbb{Z}_p\right)\subseteq{p^{-\alpha}\mathbb{Z}_p}$, which means that 

$\left(a+p^{-\beta}\mathbb{Z}_p\right)\cap{p^{-\alpha}\mathbb{Z}_p}=\left(a+p^{-\beta}\mathbb{Z}_p\right)$.

Hence $\mu\left(\left(a+p^{-\beta}\mathbb{Z}_p\right)\cap{p^{-\alpha}\mathbb{Z}_p}\right)=\mu\left(\left(a+p^{-\beta}\mathbb{Z}_p\right)\right)=\mu(p^{-\beta}\mathbb{Z}_p)=p^{\beta}=p^{v_2+v_3+v_4}$.
\item $\alpha\geq\beta$ and $\alpha<-v_p(a)$.
Let $x=a+b$ be an element in the intersection as before, hence $v_p(x)=v_p(a+b)\geq\min\{v_p(a),v_p(b)\}\geq\min\{v_p(a),-\beta\}$ and $v_p(x)\geq-\alpha$.

But $\alpha<-v_p(a)\iff{v_p(a)<-\alpha}$ and $\alpha\geq\beta\iff-\beta\geq-\alpha$, hence $v_p(x)\geq\min\{v_p(a),-\beta\}\geq\min\{v_p(a),-\alpha\}=v_p(a)$. But $v_p(a)<-\alpha$ is a strong inequality, thus by a known fact for non-Archimedean valuations $v_p(x)=\min\{v_p(a),-\alpha\}=v_p(a)$.

But all this shows that $x$ cannot be both in $p^{-\alpha}\mathbb{Z}_p$ and in $\left(a+p^{-\beta}\mathbb{Z}_p\right)$, hence $p^{-\alpha}\mathbb{Z}_p\cap\left(a+p^{-\beta}\mathbb{Z}_p\right)=\phi$.
\item $\alpha<\beta$ and $\alpha\geq{-v_p(a)}$.

Let $x=a+b$ as before. 

$v_p(x)\geq\min\{v(a),v(b)\}\geq\min\{v(a),-\beta\}\geq\min\{-\alpha,-\beta\}$.

But $-\beta<-\alpha$ is a strong inequality, thus $v(x)=\min\{-\alpha,-\beta\}=-\beta$.

On the other hand, $v(x)\geq-\alpha>-\beta$, which is a contradiction, hence $\left(a+p^{-\beta}\mathbb{Z}_p\right)\cap{p^{-\alpha}\mathbb{Z}_p}=\phi$.
\item $\alpha<\beta$ and $\alpha<-v_p(a)$.

Let $x=a+b$ as before. 

$v_p(x)\geq\min\{v(a),v(b)\}\geq\min\{v(a),-\beta\}$.

If $v_p(a)=-\beta$, then $v_p(x)\geq-\beta$, but also $v_p(x)\geq-\alpha>-\beta\Rightarrow{x\in{p^{-\alpha}\mathbb{Z}_p}}$, but all this shows that $p^{-\alpha}\mathbb{Z}_p\subseteq\left(a+p^{-\beta}\mathbb{Z}_p\right)\Rightarrow\left(a+p^{-\beta}\mathbb{Z}_p\right)\cap{p^{-\alpha}\mathbb{Z}_p}=p^{-\alpha}\mathbb{Z}_p$.

If $v_p(a)<-\beta$ then $v_p(x)=v_p(a)<-\alpha$, but that is a contradiction to $v_p(x)\geq-\alpha$.

If $v_p(a)>-\beta$ then $v(x)=-\beta<-\alpha$, which also contradicts $v_p(x)\geq-\alpha$.

Thus we have two particular cases where the intersection is not and empty set:

$\mu(b_{11})=p^{\beta}$, when $v_p(a)\geq-\alpha$ and $\alpha\geq\beta$.

$\mu(b_{11})=p^{\alpha}$, when $v_p(a)=-\beta<-\alpha\iff\alpha<\beta$.
\end{enumerate}
\newpage
Let $x=a+b\in{a}+p^{-\beta}\mathbb{Z}_p$, where $a=a_{11}a_{22}$, and $b\in{p^{-\beta}\mathbb{Z}_p}\iff{v(b)\geq-\beta}$.

Then $x\in{a+p^{-\beta}\mathbb{Z}_p}\iff{v(x)}\geq\min\{v(a),v(b)\}\geq\min\{v(a),-\beta\}$.

\begin{enumerate}
    \item case 1
    
Assume $\alpha\geq\beta\iff-\beta\geq-\alpha$.

Assume also $-v(a)\geq\alpha\iff-\alpha\geq{v(a)}$.

So, in total, $-\beta\geq-\alpha\geq{v(a)}$.

We have several options:
\begin{enumerate}
    \item $-\beta=-\alpha=v(a)$.

Then $v(x)\geq\min\{v(a),-\beta\}=v(a)=-\beta=-\alpha$, which aligns with the fact that $x\in{p^{-\alpha}\mathbb{Z}_p}$.

In conclusion, $a+p^{-\beta}\mathbb{Z}_p=p^{-\alpha}\mathbb{Z}_p$, which means that 

$\mu(\cap)=p^{\alpha}=p^{\beta}$.
    \item $-\beta\geq-\alpha>v(a)$.

Then $v(x)=\min\{v(a),-\beta\}=v(a)$.

But that is a contradiction, because $v(x)\geq-\alpha>v(a)$ (and same for $v(x)\geq-\beta\geq-\alpha>v(a)$.

\item $-\beta>-\alpha=v(a)$.

Then $v(x)=\min\{v(a),-\beta\}=v(a)=-\alpha$.

But $v(x)=v(a)=-\alpha\Rightarrow{a}+p^{-\beta}=p^{-\alpha}\mathbb{Z}_p\smallsetminus{p^{-(\alpha-1)}\mathbb{Z}_p}$ (which aligns with $v(x)\geq-\alpha\iff{x\in{p^{-\alpha}\mathbb{Z}_p}}$ only at $v(x)=-\alpha$).

So, in conclusion: 

$\left(a+p^{-\beta}\mathbb{Z}_p\right)\cap{p^{-\alpha}\mathbb{Z}_p}=p^{-\alpha}\mathbb{Z}_p\smallsetminus{p^{-(\alpha-1)}\mathbb{Z}_p}$, which means that 

$\mu(\cap)=p^{\alpha}-p^{\alpha-1}$.
\end{enumerate}
\item case 2

Assume $\alpha\geq\beta\iff-\beta\geq-\alpha$.

Assume also $-v(a)<\alpha\iff-\alpha<{v(a)}$.

We have several options:
\begin{enumerate}
    \item $v(a)>-\beta\geq-\alpha$.

Then $v(x)=\min\{v(a),-\beta\}=-\beta\geq-\alpha$, hence

$a+p^{-\beta}\mathbb{Z}_p={p^{-\beta}\mathbb{Z}_p\smallsetminus{p^{-(\beta-1)}\mathbb{Z}_p}}\subset{p^{\beta}\mathbb{Z}_p}$, which aligns with 

$v(x)\geq-\alpha\iff{x\in{p}^{-\alpha}\mathbb{Z}_p}$ (because $-\beta\geq-\alpha$).

Hence, our intersection is between the two subsets $p^{-\alpha}\mathbb{Z}_p\smallsetminus{p}^{-(\beta-1)}\mathbb{Z}_p$ and $p^{-\beta}\mathbb{Z}_p\smallsetminus{p}^{-(\beta-1)}\mathbb{Z}_p$.

Hence $\left(a+p^{-\beta}\mathbb{Z}_p\right)\cap{p^{-\alpha}\mathbb{Z}_p}={p^{-\beta}\mathbb{Z}_p\smallsetminus{p^{-(\beta-1)}\mathbb{Z}_p}}$, which means that $\mu(\cap)=p^{\beta}-p^{\beta-1}$.

    \item $-\beta\geq{v(a)}>-\alpha$.

Then $v(x)\geq\min\{v(a),-\beta\}=v(a)>-\alpha$, which also means that 

$x\in{p^{-(\alpha-1)}\mathbb{Z}_p}$ (because $v(x)>-\alpha$ is a strong inequality).

In conclusion, $a+p^{-\beta}\mathbb{Z}_p\subset{p^{-(\alpha-1)}\mathbb{Z}_p}$ 

Hence $\left(a+p^{-\beta}\mathbb{Z}_p\right)\cap{p^{-\alpha}}\mathbb{Z}_p=\left(a+p^{-\beta}\mathbb{Z}_p\right)\cap{p^{-(\alpha-1)}}\mathbb{Z}_p=\left(a+p^{-\beta}\mathbb{Z}_p\right)$, which means that $\mu(\cap)=p^{\beta}$.
\end{enumerate}
\item case 3

Assume $\beta>\alpha\iff-\alpha>-\beta$.

Assume also $-v(a)\geq\alpha\iff-\alpha\geq{v(a)}$.

We have several options:
\begin{enumerate}
    \item $-\alpha\geq{v(a)}>-\beta$.

Then $v(x)=\min\{v(a),-\beta\}=-\beta$, which is a contradiction to the fact that $x\in{p^{-\alpha}}\mathbb{Z}_p\iff{v(x)}\geq-\alpha>-\beta$.

    \item $-\alpha>{v(a)}\geq-\beta$.

Then $v(x)\geq\min\{v(a),-\beta\}=-\beta$, which aligns with the fact that ${x}\in{p^{-\alpha}}\iff{v(x)}\geq-\alpha>-\beta$.

Hence $\left(a+p^{-\beta}\mathbb{Z}_p\right)\cap{p}^{-\alpha}\mathbb{Z}_p={p}^{-\alpha}\mathbb{Z}_p$, which means that $\mu(\cap)=p^{\alpha}$.

\item $-\alpha>-\beta\geq{v(a)}$.

Then $v(x)\geq\min\{v(a),-\beta\}=v(a)$, which aligns with the fact that $x\in{p^{-\alpha}\mathbb{Z}_p}\iff{v(x)}\geq-\alpha>v(a)$.

Hence $\left(a+p^{-\beta}\mathbb{Z}_p\right)\cap{p}^{-\alpha}\mathbb{Z}_p={p}^{-\alpha}\mathbb{Z}_p$, which means that $\mu(\cap)=p^{\alpha}$.
\end{enumerate}
\item case 4

Assume $\beta>\alpha\iff-\alpha>-\beta$.

Assume also $-v(a)<\alpha\iff-\alpha<{v(a)}$.

So, in total, $v(a)>-\alpha>-\beta$.

Then $v(x)=\min\{v(a),-\beta\}=-\beta$, but this is a contradiction to the fact that $x\in{p^{-\alpha}\mathbb{Z}_p}\iff{v(x)}\geq-\alpha>-\beta$.

\end{enumerate}
So, in conclusion:

$\mu(\cap)=$

$p^{\beta}=p^{\alpha}$, if $\alpha=\beta=v(a)$.

$p^{\beta}\neq{p^{\alpha}}$, if $-\beta\geq{v(a)}>-\alpha$.

$p^{\alpha}-p^{\alpha-1}$, if $-\beta>-\alpha=v(a)$.

$p^{\beta}-p^{\beta-1}$, if $v(a)>-\beta\geq-\alpha$.

$p^{\alpha}\neq{p^{\beta}}$, $-\alpha>-\beta\geq{v(a)}$.
\end{document}
