\documentclass{article}
\usepackage{graphicx} % Required for inserting images
\usepackage{amsmath}
\usepackage{amssymb}
\title{ComputeIntegral}
\author{haiml76 }
\date{December 2025}

\begin{document}

\maketitle
In all the following, we denote by $v=v_p$ the $p$-adic Haar measure.

\section{Computation of the integral}
We want to integrate the function $|\det{M}|_p^s=|\lambda_1^4\lambda_2^6\lambda_3^6\lambda_4^4|_p^{s}$ over the monoid $G^+(\mathbb{Q}_p)$.
We denote by $v_1=v_p(\lambda_1),v_2=v_p(\lambda_2),v_3=v_p(\lambda_3),v_4=v_p(\lambda_4)$ the $p$-adic valuations of the elements on the main diagonal of $M_{11}$ the top-left block of the product matrix $M$.

We denote by $t=p^{-s}$, where $s$ is the complex parameter of our zeta-function.

Thus, $|\lambda_1\lambda_2\lambda_3\lambda_4|_p^s=|\lambda_1|_p^s|\lambda_2|_p^s|\lambda_3|_p^s|\lambda_4|_p^s=(p^{-v_1})^s(p^{-v_2})^s(p^{-v_3})^s(p^{-v_4})^s=(p^{-s})^{v_1}(p^{-s})^{v_2}(p^{-s})^{v_3}(p^{-s})^{v_4}=t^{v_1}t^{v_2}t^{v_3}t^{v_4}$.

Now we need to define our domain of integration, and calculate its Haar measure. Since we are computing a $p$-adic integral, we multiply the function to integrate by the domain measure, the result being our desired integral. 
\subsection{Computation of constraints on $c_1,c_2,c_3,c_4$}
We compute the constraints on $c_1,c_2,c_3,c_4$ by considering the following product matrix elements:
\begin{flalign*}
m_{1,10}=c_1\lambda_2\lambda_3\lambda_1\lambda_1\in\mathbb{Z}_p\iff{v(c_1\lambda_2\lambda_3\lambda_1\lambda_1)=v(c_1)+v(\lambda_1)+v(\lambda_2)+v(\lambda_3)+v(\lambda_4)\geq{0}}.&&\\
m_{2,10}=c_2\lambda_2\lambda_3\lambda_1\lambda_1\in\mathbb{Z}_p\iff{v(c_2\lambda_2\lambda_3\lambda_1\lambda_1)=v(c_2)+v(\lambda_1)+v(\lambda_2)+v(\lambda_3)+v(\lambda_4)\geq{0}}.&&\\
m_{3,10}=c_3\lambda_2\lambda_3\lambda_1\lambda_1\in\mathbb{Z}_p\iff{v(c_3\lambda_2\lambda_3\lambda_1\lambda_1)=v(c_3)+v(\lambda_1)+v(\lambda_2)+v(\lambda_3)+v(\lambda_4)\geq{0}}.&&\\
m_{4,10}=c_4\lambda_2\lambda_3\lambda_1\lambda_1\in\mathbb{Z}_p\iff{v(c_4\lambda_2\lambda_3\lambda_1\lambda_1)=v(c_4)+v(\lambda_1)+v(\lambda_2)+v(\lambda_3)+v(\lambda_4)\geq{0}}.&&
\end{flalign*}
Hence we obtain the following constraints:
\[
v(c_1),v(c_2),v(c_3),v(c_4)\geq-(v_1+v_2+v_3+v_4)\Rightarrow\mu(c_1),\mu(c_2),\mu(c_3),\mu(c_4)=p^{v_1+v_2+v_3+v_4}.\]
Hence our integral is
\[
\int_{\underline\lambda,\underline{a},\underline{b}}t^{4v_1+6v_2+6v_3+4v_4}p^{4v_1+4v_2+4v_3+4v_4}d\mu(\underline\lambda)d\mu(\underline{a})d\mu(\underline{b}).
\]
\subsection{Computation of constraints on the $b$ elements}
We consider the following product matrix elements:
\begin{flalign*}
m_{18}=b_{11}\lambda_1\lambda_2\lambda_3\in\mathbb{Z}_p\Rightarrow{v(b_{11})\geq-(v_1+v_2+v_3)}.&&\\
m_{19}=b_{12}\lambda_2\lambda_3\lambda_4\in\mathbb{Z}_p\Rightarrow{v(b_{12})\geq-(v_2+v_3+v_4)}.&&\\
m_{29}=b_{22}\lambda_2\lambda_3\lambda_4\in\mathbb{Z}_p\Rightarrow{v(b_{22})\geq-(v_2+v_3+v_4)}.&&\\
m_{38}=b_{31}\lambda_1\lambda_2\lambda_3\in\mathbb{Z}_p\Rightarrow{v(b_{31})\geq-(v_1+v_2+v_3)}.&&\\
m_{48}=b_{41}\lambda_1\lambda_2\lambda_3\in\mathbb{Z}_p\Rightarrow{v(b_{41})\geq-(v_1+v_2+v_3)}.&&\\
\end{flalign*}
We can observe also that:

\begin{flalign*}
m_{5,10}=b_{22}\lambda_1\lambda_2\lambda_3\lambda_4\in\mathbb{Z}_p\Rightarrow{v(b_{22})\geq-(v_1+v_2+v_3+v_4)}.&&\\
m_{7,10}=b_{31}\lambda_1\lambda_2\lambda_3\lambda_4\in\mathbb{Z}_p\Rightarrow{v(b_{31})\geq-(v_1+v_2+v_3+v_4)}.&&\\
\end{flalign*}
But since $v_1,v_2,v_3,v_4\geq{0}$, then $x\geq-(v_1+v_2+v_3)\implies{x}\geq-(v_1+v_2+v_3+v_4)$ and $x\geq-(v_2+v_3+v_4)\implies{x}\geq-(v_1+v_2+v_3+v_4)$, hence $m_{5,10},m_{7,10}$ do not add new constraints on $b_{22},b_{31}$.

We observe that $b_{11}$ appears also in the following expression:

$m_{49}=(a_{11}a_{22}-b_{11})\lambda_2\lambda_3\lambda_4\in\mathbb{Z}_{p}.$

Hence we have:
\begin{flalign*}
b_{11}\in{p^{-(v_1+v_2+v_3)}}.\\
a_{11}a_{22}-b_{11}\in{p^{-(v_2+v_3+v_4)}}\Rightarrow{b_{11}\in{a_{11}a_{22}+p^{-(v_2+v_3+v_4)}}}.
\end{flalign*}
Which means that $b_{11}\in{p^{-(v_1+v_2+v_3)}\mathbb{Z}_p}\cap\left(a_{11}a_{22}+p^{-(v_2+v_3+v_4)}\mathbb{Z}_p\right)$.

We compute $\mu\left(p^{-(v_1+v_2+v_3)}\mathbb{Z}_p\cap\left(a_{11}a_{22}+p^{-(v_2+v_3+v_4)}\mathbb{Z}_p\right)\right)$, 

the Haar measure of the intersection.

Denote:

$a=a_{11}a_{22}$.

$\alpha=v_1+v_2+v_3$.

$\beta=v_2+v_3+v_4$.

We have the following cases:
\begin{enumerate}
    \item $\alpha\geq\beta$ and $\alpha\geq-v_p(a)$.

Let $x\in{\left(a+p^{-\beta}\mathbb{Z}_p\right)\cap{p^{-\alpha}\mathbb{Z}_p}}$ be some element in the intersection, but if $x\in{a+p^{-\beta}\mathbb{Z}_p}$ then $x=a+b$, where $b\in{p^{-\beta}\mathbb{Z}_p}\iff{v_p(b)\geq{-\beta}}$, but then $v_p(x)\geq\min\{v_p(a),v_p(b)\}\geq\min\{v_p(a),-\beta\}$.

But $\alpha\geq-v_p(a)\iff{v_p(a)\geq-\alpha}\Rightarrow\min\{v_p(a),-\beta\}\geq\min\{-\alpha,-\beta\}$,

and $\alpha\geq\beta\iff-\beta\geq-\alpha\Rightarrow{v_p(x)\geq\min\{-\alpha,-\beta\}=-\alpha}\Rightarrow{x\in{p^{-\alpha}\mathbb{Z}_p}}$.

But all this shows that $\left(a+p^{-\beta}\mathbb{Z}_p\right)\subseteq{p^{-\alpha}\mathbb{Z}_p}$, which means that 

$\left(a+p^{-\beta}\mathbb{Z}_p\right)\cap{p^{-\alpha}\mathbb{Z}_p}=\left(a+p^{-\beta}\mathbb{Z}_p\right)$.

Hence $\mu\left(\left(a+p^{-\beta}\mathbb{Z}_p\right)\cap{p^{-\alpha}\mathbb{Z}_p}\right)=\mu\left(\left(a+p^{-\beta}\mathbb{Z}_p\right)\right)=\mu(p^{-\beta}\mathbb{Z}_p)=p^{\beta}=p^{v_2+v_3+v_4}$.
\item $\alpha\geq\beta$ and $\alpha<-v_p(a)$.
Let $x=a+b$ be an element in the intersection as before, hence $v_p(x)=v_p(a+b)\geq\min\{v_p(a),v_p(b)\}\geq\min\{v_p(a),-\beta\}$ and $v_p(x)\geq-\alpha$.

But $\alpha<-v_p(a)\iff{v_p(a)<-\alpha}$ and $\alpha\geq\beta\iff-\beta\geq-\alpha$, hence $v_p(x)\geq\min\{v_p(a),-\beta\}\geq\min\{v_p(a),-\alpha\}=v_p(a)$. But $v_p(a)<-\alpha$ is a strong inequality, thus by a known fact for non-Archimedean valuations $v_p(x)=\min\{v_p(a),-\alpha\}=v_p(a)$.

But all this shows that $x$ cannot be both in $p^{-\alpha}\mathbb{Z}_p$ and in $\left(a+p^{-\beta}\mathbb{Z}_p\right)$, hence $p^{-\alpha}\mathbb{Z}_p\cap\left(a+p^{-\beta}\mathbb{Z}_p\right)=\phi$.
\end{enumerate}
The above is symmetrical to $\beta\geq\alpha\iff-\alpha\geq-\beta$, because if $\beta\geq-v_p(a)\iff{v_p(a)}\geq-\beta$, then $v_p(x)\geq\min\{v_p(a),-\beta\}=-\beta$, but $x\in{p}^{-\alpha}\mathbb{Z}_p\iff{v(x)}\geq-\alpha\geq-\beta\implies{p}^{}$

Thus, we claim that:
\[\mu(b_{11})=
\begin{cases}
p^{\min\{\alpha,\beta\}},&\text{if}\,\,\,\,v(a)\geq-\max\{\alpha,\beta\}. \\
0,&\text{if}\,\,\,\,v(a)<-\max\{\alpha,\beta\}.
\end{cases}
\]
Hence, the Haar measure on $b_{11}$ adds the factor:

$p^{\min\{\alpha,\beta\}}=p^{\min\{v_1+v_2+v_3,v_2+v_3+v_4\}}=p^{v_2+v_3+\min\{v_1+v_4\}}$.

The measures on $b_{12},b_{22},b_{31},b_{41}$ altogether add the factor: 

$p^{2v_1}p^{4v_2}p^{4v_3}p^{2v_4}=p^{2v_1+4v_2+4v_3+2v_4}$.

Thus, our integral is now:
\[
\int_{\underline\lambda,\underline{a}}t^{4v_1+6v_2+6v_3+4v_4}p^{4v_1+4v_2+4v_3+4v_4}p^{2v_1+4v_2+4v_3+2v_4}p^{v_2+v_3+\min\{v_1+v_4\}}d\mu(\underline\lambda)d\mu(\underline{a})=\]\[=\int_{\underline\lambda,\underline{a}}t^{4v_1+6v_2+6v_3+4v_4}p^{6v_1+9v_2+9v_3+6v_4+\min\{v_1+v_4\}}d\mu(\underline\lambda)d\mu(\underline{a}).
\]
\subsection{Computation of constraints on the $a$ elements}
We have the following constraints:
\begin{flalign*}
m_{15}=a_{11}\lambda_1\lambda_2\in\mathbb{Z}_p\implies{v_p(a_{11}\lambda_1\lambda_2)\geq{0}}\implies{v_p(a_{11})}\geq-(v_1+v_2)\\
m_{25}=a_{21}\lambda_1\lambda_2\in\mathbb{Z}_p\implies{v_p(a_{21}\lambda_1\lambda_2)\geq{0}}\implies{v_p(a_{21})}\geq-(v_1+v_2)\\
m_{26}=a_{22}\lambda_2\lambda_3\in\mathbb{Z}_p\implies{v_p(a_{22}\lambda_1\lambda_2)\geq{0}}\implies{v_p(a_{22})}\geq-(v_2+v_3)\\
m_{36}=-a_{11}\lambda_2\lambda_3\in\mathbb{Z}_p\implies{v_p(a_{11}\lambda_1\lambda_2)\geq{0}}\implies{v_p(a_{11})}\geq-(v_2+v_3)\\
m_{37}=a_{33}\lambda_3\lambda_4\in\mathbb{Z}_p\implies{v_p(a_{33}v_1v_2)\geq{0}}\implies{v_p(a_{33})}\geq-(v_3+v_4)\\
m_{47}=-a_{22}\lambda_3\lambda_4\in\mathbb{Z}_p\implies{v_p(a_{22}\lambda_1\lambda_2)\geq{0}}\implies{v_p(a_{22})}\geq-(v_3+v_4)\\
\end{flalign*}
We observe that there are more constraints coming from products of $a$ and $\lambda$ elements, but they are contained in the constraints we listed above.
In addition, we have the following constraints:
\begin{flalign*}
m_{28}=a_{21}a_{22}\lambda_1\lambda_2\lambda_3\in\mathbb{Z}_p\implies{v_p(a_{21}a_{22}\lambda_1\lambda_2\lambda_3)\geq{0}}\implies{v_p(a_{22})+v_p(a_{21})}\geq-(v_1+v_2+v_3)\\
m_{39}=-a_{11}a_{33}\lambda_2\lambda_3\lambda_4\in\mathbb{Z}_p\implies{v_p(a_{11}a_{33}\lambda_2\lambda_3\lambda_4)\geq{0}}\implies{v_p(a_{11})+v_p(a_{33})}\geq-(v_2+v_3+v_4)\\
m_{6,10}=-a_{21}a_{33}\lambda_1\lambda_2\lambda_3\lambda_4\in\mathbb{Z}_p\implies{v_p(a_{21}a_{33}\lambda_1\lambda_2\lambda_3\lambda_4)\geq{0}}\implies{v_p(a_{21})+v_p(a_{33})}\geq-(v_1+v_2+v_3+v_4)\\
\end{flalign*}
The two constraints on $a_{33}$ are not contained in each other, because 

$v_p(a_{33})\geq-v_p(a_{11})-v_2-v_3-v_4$, hence:

$-v_p(a_{11})-v_2-v_3-v_4\geq{-v_3-v_4}\iff{-v_p(a_{11})-v_2}\geq{0}\iff{v_p(a_{11})\leq-v_2\leq{0}}$.

$-v_p(a_{11})-v_2-v_3-v_4<{-v_3-v_4}\iff{-v_p(a_{11})-v_2}<{0}\iff{v_p(a_{11})>-v_2}$.

And hence, $v_{33}\geq-(v_3+v_4)-\min\{0,v_2+v_p(a_{11})\}$.

By considering the two constraints on $a_{21}$ exactly the same way, we conclude that $v_p(a_{21})\geq-(v_1+v_2)-\min\{0,v_3+v_p(a_{22})\}$.


\end{document}
