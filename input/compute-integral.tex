\documentclass{article}
\usepackage{graphicx} % Required for inserting images
\usepackage{amsmath}
\usepackage{amssymb}
\title{ComputeIntegral}
\author{haiml76 }
\date{December 2025}

\begin{document}

\maketitle
In all the following, we denote by $v=v_p$ the $p$-adic Haar measure.
\section{Computation of $c_1,c_2,c_3,c_4$}
The elements $c_1,c_2,c_3,c_4$ are located in block $N_{14}$, hence they appear in the following product matrix elements:
\begin{flalign*}
m_{1,10}=c_1\lambda_2\lambda_3\lambda_1\lambda_1\in\mathbb{Z}_p\Rightarrow{v(c_1\lambda_2\lambda_3\lambda_1\lambda_1)=v(c_1)+v(\lambda_1)+v(\lambda_2)+v(\lambda_3)+v(\lambda_4)\geq{0}}.&&\\
m_{2,10}=c_2\lambda_2\lambda_3\lambda_1\lambda_1\in\mathbb{Z}_p\Rightarrow{v(c_2\lambda_2\lambda_3\lambda_1\lambda_1)=v(c_2)+v(\lambda_1)+v(\lambda_2)+v(\lambda_3)+v(\lambda_4)\geq{0}}.&&\\
m_{3,10}=c_3\lambda_2\lambda_3\lambda_1\lambda_1\in\mathbb{Z}_p\Rightarrow{v(c_3\lambda_2\lambda_3\lambda_1\lambda_1)=v(c_3)+v(\lambda_1)+v(\lambda_2)+v(\lambda_3)+v(\lambda_4)\geq{0}}.&&\\
m_{4,10}=c_4\lambda_2\lambda_3\lambda_1\lambda_1\in\mathbb{Z}_p\Rightarrow{v(c_4\lambda_2\lambda_3\lambda_1\lambda_1)=v(c_4)+v(\lambda_1)+v(\lambda_2)+v(\lambda_3)+v(\lambda_4)\geq{0}}.&&
\end{flalign*}
We denote $v_i=v(\lambda_i).$ From the constraints above we obtain:
\[
v(c_1),v(c_2),v(c_3),v(c_4)\geq-(v_1+v_2+v_3+v_4)\Rightarrow\mu(c_1),\mu(c_2),\mu(c_3),\mu(c_4)=p^{v_1+v_2+v_3+v_4}.\]





Assume $|a|\neq|b|,$ and assume wlog $|a|<|b|$

The absolute value is non-Archimedean, that is:
\[|a+b|\leq\max\{|a|,|b|\}=|b|.\]

But then:

$|b|=|a+b-a|=|(a+b)+(-a)|\leq\max\{|a+b|,|-a|\}=\max\{|a+b|,|a|\}.$

We assumed $|a|<|b|$, so $|b|\leq\max\{|a+b|,|a|\}$ only if $|b|\leq|a+b|$.

But then:
\[|b|\leq|a+b|\]
\[|a+b|\leq|b|\]
\[\Rightarrow|a+b|=|b|.\]
\end{document}
