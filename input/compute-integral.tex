\documentclass{article}
\usepackage{graphicx} % Required for inserting images
\usepackage{amsmath}
\usepackage{amssymb}
\title{ComputeIntegral}
\author{haiml76 }
\date{December 2025}

\begin{document}

\maketitle
In all the following, we denote by $v=v_p$ the $p$-adic Haar measure.

\section{Computation of the integral}
We want to integrate the function $|\det{M}|_p^s=|\lambda_1^4\lambda_2^6\lambda_3^6\lambda_4^4|_p^{s}$ over the monoid $G^+(\mathbb{Q}_p)$.
We denote by $v_1=v_p(\lambda_1),v_2=v_p(\lambda_2),v_3=v_p(\lambda_3),v_4=v_p(\lambda_4)$ the $p$-adic valuations of the elements on the main diagonal of $M_{11}$ the top-left block of the product matrix $M$.

We denote by $t=p^{-s}$, where $s$ is the complex parameter of our zeta-function.

Thus, $|\lambda_1\lambda_2\lambda_3\lambda_4|_p^s=|\lambda_1|_p^s|\lambda_2|_p^s|\lambda_3|_p^s|\lambda_4|_p^s=(p^{-v_1})^s(p^{-v_2})^s(p^{-v_3})^s(p^{-v_4})^s=(p^{-s})^{v_1}(p^{-s})^{v_2}(p^{-s})^{v_3}(p^{-s})^{v_4}=t^{v_1}t^{v_2}t^{v_3}t^{v_4}$.

Now we need to define our domain of integration, and calculate its Haar measure. Since we are computing a $p$-adic integral, we multiply the function to integrate by the domain measure, the result being our desired integral. 
\subsection{Computation of $c_1,c_2,c_3,c_4$}
The elements $c_1,c_2,c_3,c_4$ are located in block $N_{14}$, hence they appear in the following product matrix elements:
\begin{flalign*}
m_{1,10}=c_1\lambda_2\lambda_3\lambda_1\lambda_1\in\mathbb{Z}_p\Rightarrow{v(c_1\lambda_2\lambda_3\lambda_1\lambda_1)=v(c_1)+v(\lambda_1)+v(\lambda_2)+v(\lambda_3)+v(\lambda_4)\geq{0}}.&&\\
m_{2,10}=c_2\lambda_2\lambda_3\lambda_1\lambda_1\in\mathbb{Z}_p\Rightarrow{v(c_2\lambda_2\lambda_3\lambda_1\lambda_1)=v(c_2)+v(\lambda_1)+v(\lambda_2)+v(\lambda_3)+v(\lambda_4)\geq{0}}.&&\\
m_{3,10}=c_3\lambda_2\lambda_3\lambda_1\lambda_1\in\mathbb{Z}_p\Rightarrow{v(c_3\lambda_2\lambda_3\lambda_1\lambda_1)=v(c_3)+v(\lambda_1)+v(\lambda_2)+v(\lambda_3)+v(\lambda_4)\geq{0}}.&&\\
m_{4,10}=c_4\lambda_2\lambda_3\lambda_1\lambda_1\in\mathbb{Z}_p\Rightarrow{v(c_4\lambda_2\lambda_3\lambda_1\lambda_1)=v(c_4)+v(\lambda_1)+v(\lambda_2)+v(\lambda_3)+v(\lambda_4)\geq{0}}.&&
\end{flalign*}
We denote $v_i=v(\lambda_i).$ From the constraints above we obtain:
\[
v(c_1),v(c_2),v(c_3),v(c_4)\geq-(v_1+v_2+v_3+v_4)\Rightarrow\mu(c_1),\mu(c_2),\mu(c_3),\mu(c_4)=p^{v_1+v_2+v_3+v_4}.\]
Hence our integral is
\[
\int_{\underline\lambda,\underline{a},\underline{b}}t^{4v_1+6v_2+6v_3+4v_4}p^{4v_1+4v_2+4v_3+4v_4}d\mu(\underline\lambda)d\mu(\underline{a})d\mu(\underline{b}).
\]
\subsection{Computation of the elements in $N_{23}.$}
We have the following constraints:
\begin{equation}
m_{58}=a_{22}\lambda_1\lambda_2\lambda_3\in\mathbb{Z}_p\Rightarrow{v(a_{22})\geq-(v_1+v_2+v_3)}.\\
\end{equation}
\begin{equation}
m_{68}=a_{21}\lambda_1\lambda_2\lambda_3\in\mathbb{Z}_p\Rightarrow{v(a_{21})\geq-(v_1+v_2+v_3)}.\\
\end{equation}
\begin{equation}
m_{69}=a_{33}\lambda_2\lambda_3\lambda_4\in\mathbb{Z}_p\Rightarrow{v(a_{33})\geq-(v_2+v_3+v_4)}.\\
\end{equation}
\begin{equation}
m_{79}=-a_{11}\lambda_2\lambda_3\lambda_4\in\mathbb{Z}_p\Rightarrow{v(-a_{11})=v(a_{11})\geq-(v_2+v_3+v_4)}.
\end{equation}
We observe that all the elements in this block appear also in block $N_{12}$, but the constraints on them obtained from block $N_{12}$ are contained in the constraints obtained from block $N_{23}.$
\subsection{}
\begin{flalign*}
m_{18}=b_{11}\lambda_1\lambda_2\lambda_3\in\mathbb{Z}_p\Rightarrow{v(b_{11})\geq-(v_1+v_2+v_3)}.&&\\
m_{19}=b_{12}\lambda_2\lambda_3\lambda_4\in\mathbb{Z}_p\Rightarrow{v(b_{12})\geq-(v_2+v_3+v_4)}.&&\\
m_{29}=b_{22}\lambda_2\lambda_3\lambda_4\in\mathbb{Z}_p\Rightarrow{v(b_{22})\geq-(v_2+v_3+v_4)}.&&\\
m_{38}=b_{31}\lambda_1\lambda_2\lambda_3\in\mathbb{Z}_p\Rightarrow{v(b_{31})\geq-(v_2+v_3+v_4)}.&&\\
m_{48}=b_{41}\lambda_1\lambda_2\lambda_3\in\mathbb{Z}_p\Rightarrow{v(b_{41})\geq-(v_2+v_3+v_4)}.&&\\
\end{flalign*}

\end{document}
